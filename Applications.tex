\section*{WDA - Wasserstein Discriminant Analysis}

Rémi Flamary, Marco Cuturi, Nicolas Courty, Alain Rakotomamonjy
(Submitted on 29 Aug 2016 (v1), last revised 23 May 2018 (this version, v2))
Wasserstein Discriminant Analysis (WDA) is a new supervised method that can improve classification of high-dimensional data by computing a suitable linear map onto a lower dimensional subspace. Following the blueprint of classical Linear Discriminant Analysis (LDA), WDA selects the projection matrix that maximizes the ratio of two quantities: the dispersion of projected points coming from different classes, divided by the dispersion of projected points coming from the same class. To quantify dispersion, WDA uses regularized Wasserstein distances, rather than cross-variance measures which have been usually considered, notably in LDA. Thanks to the underlying principles of optimal transport, WDA is able to capture both global (at distribution scale) and local (at samples scale) interactions between classes. Regularized Wasserstein distances can be computed using the Sinkhorn matrix scaling algorithm; We show that the optimization of WDA can be tackled using automatic differentiation of Sinkhorn iterations. Numerical experiments show promising results both in terms of prediction and visualization on toy examples and real life datasets such as MNIST and on deep features obtained from a subset of the Caltech dataset. 