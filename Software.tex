
\chapter{Appendix Numerical Software}
Computers are getting more powerful over time but size of the problems we're solving scales with the increased performance. Tools for acquiring and storing data are improving at an even faster pace than processors.  It turns out the communication is the real bottleneck to scaling many algorithms. The capacity of fast memory close to the computing resource (cache) grows very slowly in time.  Commodity hardware is now being used to deploy large distributed computing systems running Apache Hadoop and Spark. The possibility for computing on ever larger data sets is tantalizing for the machine learning community.
These notes discuss the a number of numerical computing libraries the author has used in various machine learning and image processing projects.  There is a lot of effort underway to integrate and distribute some of the algorithms in these libraries to the Hadooop ecosystem.

Many of the libraries discussed here are available in Matlab. Intel MLK \& IPP, Arpack, UMFPACK, and SDPA have been integrated in the autor's klMatrix library see the section on the KL Libraries below for specifics.

\section*{BLAS MLK \& Atlas}

\section*{Graphviz - Graph Visualization Software}
Graphviz ( http://www.graphviz.org ) is open source graph visualization software. Graph visualization is a way of representing structural information as diagrams of abstract graphs and networks. It has important applications in networking, bioinformatics,  software engineering, database and web design, machine learning, and in visual interfaces for other technical domains.

Graphviz is used to generate collaboration, inheritance, and call diagrams the KL documentation. There is an API that is used in the KL framework to facilitate graph visualization.

\section*{ARPACK}
ARPACK++ is an object-oriented version of the Fortran ARPACK package. ARPACK is designed to compute a few eigenvalues and eigenvectors of large scale sparse matrices and pencils via the Arnoldi process for finding eigenvalues called. These methods utilize Krylov Subspace Projections for iterative solution that avoids matrix multiplication.  ARPACK implements the implicit restarted Arnoldi method which reduces the storage requirements of the traditional Lanczos iteration for Hermitian matrices and Arnoldi iteration for general matrices.  The key to the Krylov method is to calculate the linear subspace of $\Real^{(n,n)}$ induced by span of the first m powers of the image of $b$ under a linear operator $A$, $\kappa_m(A,b) | A \in \mathbb R^{(n,n)}
b\ in \mathbb R^n = \{b, Ab (A)^2b, \ldots (A)^mb \}$.  This avoids direct matrix matrix operations when finding the first few eigenvector, eigenvalue pairs in a large system of linear
equations.

\section*{ATLAS}
Automatically Tuned Linear Algebra software.

\section*{METIS}
METIS is a software library for finite element analysis and graph partitions.  It also can be used to reduce the fill order of
sparse matrices.

\section*{SDPA}
SDPA is a software library for solving SDPs using on the Mehrotra-type predictor-corrector infeasible primal-dual interior-point method. It is implemented C++ language and utilizes the machine dependent BLAS such as Intel MKL, ATLAS. LAPACK routines are used for matrix computations.  Efficient methods to compute the search directions exploiting the sparsity of the data matrices are implemented. Sparse or dense Cholesky factorization for the Schur complemetn matrix is automatically selected. The calculation of the Schur complement
matrix is implemented in reentrant code. A sparse version of SDPA is available that uses METIS and SPOOLES libraries for finding a proper sparse structure of the problem.

\section*{SPOOLS}
SPOOLES is a library for solving sparse real and complex linear systems of equations. SPOOLES can factor and solve square linear systems of equations with symmetric structure, and it can compute multiple minimum degree, generalized nested dissection and multisection orderings of matrices with symmetric structure.  SPOOLES utilizes a variety of Krylov iterative methods. The preconditioner is a drop tolerance factorization.

\section*{SuperLU}
SuperLU ( http://crd-legacy.lbl.gov/~xiaoye/SuperLU/) is a general purpose library for the direct solution of large, sparse, nonsymmetric systems of linear equations on high performance machines. The library is written in C and is callable from either C or Fortran. The library routines will perform an LU decomposition with partial pivoting and triangular system solves through forward and back substitution. The LU factorization routines can handle non-square matrices but the triangular solves are performed only for square matrices. The matrix columns may be preordered (before factorization) either through library or user supplied routines. This pre-ordering for sparsity is completely separate from the factorization. Working precision iterative refinement subroutines are provided for improved backward stability. Routines are also provided to equilibrate the system, estimate the condition number, calculate the relative backward error, and estimate error bounds for the refined solutions.

\section*{SuiteSparse}
Tim Davis' ( http://www.cise.ufl.edu/~davis/welcome.html) collection of sparse matrix software.  Tim is also the curator of The University of Florida Sparse Matrix Collection (http://www.cise.ufl.edu/research/sparse/matrices/) a must see for anyone interested in sparse
matrices and visualization.

AMD: symmetric approximate minimum degree
BTF: permutation to block triangular form
CAMD: symmetric approximate minimum degree
CCOLAMD: constrained column approximate minimum degree
COLAMD: column approximate minimum degree
CHOLMOD: sparse supernodal Cholesky factorization and update/downdate
CSparse: a concise sparse matrix package
CXSparse: an extended version of CSparse
KLU: sparse$ LU$ factorization, for circuit simulation
LDL: a simple $LDL^T$ factorization
UMFPACK: sparse multifrontal $LU$ factorization
RBio: MATLAB toolbox for reading/writing sparse matrices
UFconfig: common configuration for all but CSparse
SuiteSparseQR: multifrontal sparse $QR$

\section*{SuiteSparse AMD}
AMD is a set of routines for pre-ordering a sparse matrix prior to numerical factorization. It uses an approximate minimum degree ordering algorithm to find a permutation matrix P so that the Cholesky factorization $PAP^\dag =LL^\dag$ has fewer (often much fewer) nonzero entries than the Cholesky factorization of A. The algorithm is typically much faster than other ordering methods and minimum degree ordering algorithms that compute an exact degree . Some methods, such as approximate deficiency [Rothberg and Eisenstat 1998] and graph-partitioning based methods [Hendrickson and Rothberg 1999; Karypis and Kumar 1998; Pellegrini et al. 2000; Schulze 2001] can produce better orderings, depending on the matrix. The algorithm starts with an undirected graph representation of a symmetric sparse matrix . Node $i$ in the graph corresponds to row and column i of the matrix, and there is an edge $(i,j)$ in the graph if $a_{ij}$ is nonzero. The degree of a node is initialized to the number of off diagonal non-zeros in row $i$, which is the size of the set of nodes adjacent to $i$ in the graph.

\section*{SuiteSparse UMFPACK}
UMFPACK is a set of routines for solving systems of linear equations, $Ax = b$, when $A$ is sparse and unsymmetric. It is based on the Unsymmetric-pattern MultiFrontal method. UMFPACK factorizes $PAQ$, $PRAQ$ and $PR^{-1}AQ$, into the product $LU$, where $L$ and $U$ are lower and upper triangular, respectively, $P$ and $Q$ are permutation matrices, and $R$ is a diagonal matrix of row scaling factors (or $R = I$ if row-scaling is not used). Both $P$ and $Q$ are chosen to reduce fill-in (new nonzeros in $L$ and $U$ that are not present in $A$). The permutation $P$ has the dual role of reducing fill-in and maintaining numerical accuracy (via relaxed partial pivoting and row interchanges). The sparse matrix $A$ can be square or rectangular, singular or non-singular, and real or complex (or any combination). Only square matrices $A$ can be used to solve $Ax = b$ or related systems. Rectangular matrices can only be factorized.

\section*{fftw}
fftw is a highly optimized library for calculating the discrete Fourier transform. \cite{G_ieeetrans_FFTW}.  Generally fftw performs better than Intel IPP.  fftw can be sped up by compiling with the Intel Compiler.

\section*{Simulation and modeling with the kl Software
Framework}Class, interaction and collaboration diagrams are presented below for a modeling framework implemented by the author. The framework is implemented in C++.  The simulation of various univariate and multivariate random number generators along with the distribution tests from the CDHC library are included as well.
Features of this framework include:
\begin{itemize}
\item utilizing optimized BLAS libraries
\item the up to date methods for univariate random number generation
\item wrappers for Intel performance primitives and GSL
\item multiple memory management facilities
\end{itemize}
We will use the term $EPDF_X$ to mean the empirical probability density function. There are a variety of univariate tests to help determine which parametric distribution your data belongs to. These fall under the category of Goodness of Fit testing. For a parametric family the null hypothesis $H_o : X=_d p(x| \theta)$ is tested against the alternative that $X$ does not belong to the family $p(x|\theta)$ There are also family of test to determine whether two $EPDF$'s come from the same distribution. 